%!TEX TS-program = pdflatex
%!BIB program = biber
% -- Author: Jannes Bantje, j.bantje@wwu.de
%!TEX root = bachelor.tex
% -- Author: Jannes Bantje, j.bantje@wwu.de
\documentclass[a4paper,index=totoc,toc=bibliography,fontsize=10,DIV=13,headinclude,twoside,BCOR=12mm,cleardoublepage=empty,draft]{scrreprt}




%-- typografische Verbesserungen, Codierungskram, Schriftwahl und erste Mathepakete
\usepackage[utf8]{inputenc}
\usepackage[lining,semibold,defaultfeatures={Variant=01}]{libertine}
\usepackage[T1]{fontenc}
\usepackage{textcomp} % verhindert ein paar Fehler bei den Fonts
\usepackage[varl]{inconsolata}
\usepackage{mathtools,amssymb,amsthm} % Verbesserung von amsmath (die amsmath selbst lädt)
\usepackage[libertine,cmintegrals,bigdelims,varbb]{newtxmath}
\usepackage[ngerman]{babel}
\usepackage[babel=true, tracking=true,final]{microtype}
% \useosf % aktiviert sog. "old style figures", also werden Zahlen – im Text – teilweise unterhalb der Grundlinie angezeigt. Muss man mögen...

%-- Basics für graphische Sachen
\usepackage[usenames,x11names]{xcolor} % Die Optionen definieren zusätzliche Farben (siehe Dokumentation)
\usepackage[final]{graphicx}

%-- Mathematikpakete und Einstellungen
\mathtoolsset{centercolon} % sorgt dafür dass := und =: besser aussehen
\usepackage{mathdots} % sorgt dafür, dass Punte wie zB \ddots besser aussehen
\newcommand{\Underbrace}[2]{{\underbrace{#1}_{#2}}} % Underbrace als Befehl in LaTeX-Syntax (und ohne Spacing-Probleme mit nachfolgenden Operatoren...)
\renewcommand{\le}{\leqslant} % ich finde Kleinergleich mit schrägen Strich schöner
\renewcommand{\ge}{\geqslant}

%-- kommutative Diagramme
\usepackage{tikz-cd} %-- meiner Meinung nach das beste Paket für kommutative Diagramme
\tikzset{% um Kompatibilität mit Babel herzustellen und die angenehme "<label>"-Syntax zu nutzen
  every picture/.append style={
    execute at begin picture={\shorthandoff{"}},
    execute at end picture={\shorthandon{"}}
  }
}
\usetikzlibrary{quotes,babel}

%-- Für Literaturangaben, hier wird NICHT das total veraltete bibtex benutzt!
\usepackage[%
	backend=biber,
	sortlocale=auto,
	natbib,
	hyperref,
	backref,
	style=alphabetic % eine unvollständige Auswahl von Styles: ieee, numeric, apa
	]%
{biblatex}
\addbibresource{quellen.bib} % Literaturdatei einlesen

% -- Konfiguration von Hyperref (sorgt für anklickbare Links und ein PDF-Inhaltsverzeichnis)
\usepackage[hidelinks, pdfpagelabels, bookmarksopen=true, bookmarksnumbered=true, linkcolor=black, urlcolor=SkyBlue2, plainpages=false,pagebackref, citecolor=black, hypertexnames=true, pdfauthor={Jannes Bantje}, pdfborderstyle={/S/U}, linkbordercolor=SkyBlue2, colorlinks=false,backref=false]{hyperref}
\hypersetup{final}

%-- Für Aufzählungen und andere Listen, Anführungszeichen und Zitate
\usepackage[shortlabels]{enumitem} % durch die Option ist die gleiche Syntax wie zB mit dem Paket paralist möglich
\setlist[enumerate,description]{font=\sffamily\bfseries} % sorgt dafür, dass die Labels bei enumerate und description fett sind
\usepackage[german=quotes]{csquotes}

%-- Für hilfreiche Anmerkungen am Seitenrand
\usepackage[obeyDraft,textsize=small]{todonotes}

%-- Kopf- und Fußzeilen bearbeiten
\usepackage{scrpage2}
\pagestyle{scrheadings}
\clearscrheadfoot % Standardkonfiguration löschen
\setheadsepline{1pt} % Linie für die Kopfzeile
\automark[section]{chapter} % definiert, welcher Text in den Kolumnentiteln erscheinen soll
\rohead{\rightmark} % section erscheint rechts oben
\lehead{\scshape\leftmark} % chapter erscheint links oben in ist in small caps gesetzt
\ofoot[\pagemark]{\pagemark} % Seitenzahlen immer außen, hier wir auch der plain Stil bearbeitet!
\ifoot[Titel der Bachelorarbeit]{Titel der Bachelorarbeit}
\renewcommand*{\pnumfont}{\LARGE\sffamily} % Seitenzahlen in groß und serifenlos
\renewcommand*{\footfont}{\large\sffamily\color{gray}}

%-- Theorem-Pakete und Konfiguration
\usepackage{thmtools}

\declaretheoremstyle[%
	headfont=\sffamily\bfseries,
	notefont=\normalfont\sffamily,
	bodyfont=\normalfont,
	headformat=\NUMBER\ \NAME\NOTE,
	headpunct={},
	postheadspace=1ex,
	spaceabove=15pt,spacebelow=10pt,]%
{mainstyle}
\declaretheoremstyle[%
	headfont=\normalfont\scshape,
	bodyfont=\normalfont,
	headpunct=:,
	postheadspace=1ex,
	spacebelow=12pt,spaceabove=2pt,
	qed=\qedsymbol]%
{beweise}

\declaretheorem[name=Definition,parent=section,style=mainstyle]{definition}
\declaretheorem[name=Satz,sharenumber=definition,style=mainstyle]{satz}
\declaretheorem[name=Korollar,sharenumber=definition,style=mainstyle]{korollar}
\declaretheorem[name=Lemma,sharenumber=definition,style=mainstyle]{lemma}
\declaretheorem[name=Proposition,sharenumber=definition,style=mainstyle]{proposition}

\declaretheorem[name=Beweis,numbered=no,style=beweise]{beweis}

\begin{document}
\pagenumbering{Roman} % Seitennummerierung auf römische Zahlen setzen
\begin{titlepage}
	% Nach einer Vorlage von http://www.LaTeXTemplates.com
	\newcommand{\HRule}{\rule{\linewidth}{0.5mm}} % Defines a new command for the horizontal lines, change thickness here

	\center % Center everything on the page
 
	%----------------------------------------------------------------------------------------
	%	HEADING SECTIONS
	%----------------------------------------------------------------------------------------
	
	% \includegraphics[height=1.1cm,keepaspectratio]{Bilder/Logo_WWU_Muenster.pdf} \hfill 

	\textsc{\LARGE Westfälische Wilhelms-Universität Münster}\\[1.5cm] % Name of your university/college
	\textsc{\Large Bachelorarbeit}\\[0.5cm] % Major heading such as course name

	%----------------------------------------------------------------------------------------
	%	TITLE SECTION
	%----------------------------------------------------------------------------------------

	\HRule \\[0.4cm]
	{ \huge \sffamily\bfseries Thema der Bachelorarbeit}\\[0.4cm] % Title of your document
	\HRule \\[1.5cm]
 
	%----------------------------------------------------------------------------------------
	%	AUTHOR SECTION
	%----------------------------------------------------------------------------------------

	\begin{minipage}{0.4\textwidth}
	\begin{flushleft} \large
	\emph{Author:}\\
	John \textsc{Smith}\\ % Your name
	\normalsize \url{mail@adresse}\\
	Matr.\,Nr. 123456
	\end{flushleft}
	\end{minipage}
	~
	\begin{minipage}{0.4\textwidth}
	\begin{flushright} \large
	\emph{Betreuer:} \\
	Dr. James \textsc{Smith}\\ % Supervisor's Name
	\normalsize \url{mail@adresse}
	\end{flushright}
	\end{minipage}\\[4cm]

	% If you don't want a supervisor, uncomment the two lines below and remove the section above
	%\Large \emph{Author:}\\
	%John \textsc{Smith}\\[3cm] % Your name

	%----------------------------------------------------------------------------------------
	%	DATE SECTION
	%----------------------------------------------------------------------------------------

	{\large eingereicht am \today}\\[3cm] % Date, change the \today to a set date if you want to be precise

	%----------------------------------------------------------------------------------------
	%	LOGO SECTION
	%----------------------------------------------------------------------------------------

	\includegraphics[height=1.3cm,keepaspectratio]{Bilder/fb10logo.pdf}\\[1cm] % Include a department/university logo - this will require the graphicx package
 
	%----------------------------------------------------------------------------------------

	\vfill % Fill the rest of the page with whitespace
	
\end{titlepage}
\begin{abstract}
\section*{Vorwort}
Hier entsteht ein Vorwort.
\end{abstract}
\tableofcontents

\[
	\sum_{i=0}^{\infty} a^i = \int y^2 \,\mathrm{d} x 
\]

\section{Demobereich} % (fold)
\subsection{Demonstration von \texttt{todonotes}} 
Man kann mittels \texttt{\textbackslash{}todo} Notizen an den Rand schreiben.\todo{Hier muss noch was hinzugefügt werden \ldots} Dafür muss allerdings in dem \texttt{\textbackslash{}documentclass}-Befehl die Option \texttt{draft} gesetzt sein. Man kann sich auch eine große Box mitten in den Text setzen lassen, die eine noch anzufertigende Zeichnung markiert. Dazu benutzt man \texttt{\textbackslash{}missingfigure$\{$\ldots$\}$}:
\missingfigure{Hier fehlt eine Zeichnung}

\subsection{Demonstration von \texttt{tikzcd}}
Mit der Umgebung \texttt{tikzcd} kann man sehr einfach schöne kommutative Diagramme zeichnen, die man auch noch bis aufs Feinste konfigurieren kann.
\[
	\begin{tikzcd}[sep=large]
		A \rar["f"] \dar[dashed] & B \dar[hook] \\
		C \urar["g"] & D
	\end{tikzcd}
\]
Kleiner Vergleich von \texttt{\textbackslash{}underbrace} und \texttt{\textbackslash{}Underbrace}
\[
	\underbrace{a+b}_{=1} + (x-y) \qquad \text{versus} \qquad \Underbrace{a+b}{=1} + (x-y)
\]

\subsection{Literaturangaben und Zitate} % (fold)
Wichtige Info vorweg: Das Aussehen der Referenzen hängt sehr stark von dem verwendeten Stil ab! Der einfachste Befehl zum Zitieren ist 
\texttt{\textbackslash{}cite}, dessen Output in etwas das Folgende ist: \cite{CGL}. Man kann noch ein optionales Argument angeben und da zum Beispiel das Kapitel reinschreiben: \cite[Kapitel $\pi$]{CGL}.
\begin{itemize}
	\item \textquote[\cite{CGL}]{kurzes Zitat von Gauß} erzeugt mit \texttt{\textbackslash{}textcite$[\backslash\texttt{cite}\{\ldots\}]\{\ldots\}$}
	\item Für längere Zitat benutzt man besser den Befehl \texttt{\textbackslash{}blockcite$[\backslash\texttt{cite}\{\ldots\}]\{\ldots\}$}, der je nach 
	Länge des Arguments den zitierten Text vom Fließtext absetzt: \blockquote[\cite{CGL}]{Hier kommt jetzt ein ganz langer Text hin, von dem ich hoffe, 
	dass ihn keiner liest, denn er enthält nun mal so gerade gar keine sinnvolle Information. Hier kommt jetzt ein ganz langer Text hin, von dem ich hoffe, 
	dass ihn keiner liest, denn er enthält nun mal so gerade gar keine sinnvolle Information. Hier kommt jetzt ein ganz langer Text hin, von dem ich hoffe, 
	dass ihn keiner liest, denn er enthält nun mal so gerade gar keine sinnvolle Information.}
	Hier geht der Fließtext weiter.
	\item Möchte man einfach nur ein Wort mit Anführungszeichen versehen benutzt man \texttt{\textbackslash{}enquote$\{\ldots \}$}. Zu Beispiel so:
	Die sogenannte \enquote{Lie-Algebra}.
\end{itemize}

% subsection literaturangaben_und_zitate (end)
% section demobereich (end)

\printbibliography
\end{document}
